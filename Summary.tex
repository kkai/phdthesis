\chapter{Thesis Overview}


%for citations
\begin{flushright} 
\textit{``An approximate answer to the 
right problem is worth a good deal more 
than an exact answer to an approximate problem. \\
- John Tukey}
\end{flushright} 
\begingroup
    %\vspace*{\beforechapskip}% 
    %\smash{\rule{2.6pt}{25mm}}
\textit{This chapter presents an outline of the 
thesis, to give the reader guidance what is covered 
and how to read this thesis. All publications that
are part of this thesis are outlined, and dependencies
regarding sections etc. are given.}
\\
\vskip\onelineskip
\begin{adjustwidth}{}{-\chapindent}%
\hrulefill   
\end{adjustwidth}\endgroup
\vskip\onelineskip



\begin{figure}[t]
    \begin{center}
    \includegraphics[height=2.5in]{thesis-overview.pdf}
	\end{center}
    \caption[]{Wifi signal strength depending on the room . }
\label{fig:sigstrength1}
\end{figure}




K.~Kunze and P.~Lukowicz.
\newblock Using acceleration signatures from everyday activities for on-body
  device location.
\newblock {\em Wearable Computers, 2007 11th IEEE International Symposium on},
  pages 115 -- 116, Sep 2007.


K.~Kunze and P.~Lukowicz.
\newblock Dealing with sensor displacement in motion-based onbody activity
  recognition systems.
\newblock {\em UbiComp '08: Proceedings of the 10th international conference on
  Ubiquitous computing}, Sep 2008.


K.~Kunze, P.~Lukowicz, H.~Junker, and G.~Troester.
\newblock Where am i: Recognizing on-body positions of wearable sensors.
\newblock {\em LOCA'04: International Workshop on Locationand Context-
  {\ldots}}, Jan 2005.


K.~Kunze, P.~Lukowicz, and K.~Partridge.  \newblock Which way am i
facing: Inferring horizontal device orientation from an accelerometer
signal.  \newblock {International Symposium on Wearable Computing},
Jan 2009.
